\documentclass[]{article}
\usepackage{lmodern}
\usepackage{amssymb,amsmath}
\usepackage{ifxetex,ifluatex}
\usepackage{fixltx2e} % provides \textsubscript
\ifnum 0\ifxetex 1\fi\ifluatex 1\fi=0 % if pdftex
  \usepackage[T1]{fontenc}
  \usepackage[utf8]{inputenc}
\else % if luatex or xelatex
  \ifxetex
    \usepackage{mathspec}
  \else
    \usepackage{fontspec}
  \fi
  \defaultfontfeatures{Ligatures=TeX,Scale=MatchLowercase}
\fi
% use upquote if available, for straight quotes in verbatim environments
\IfFileExists{upquote.sty}{\usepackage{upquote}}{}
% use microtype if available
\IfFileExists{microtype.sty}{%
\usepackage[]{microtype}
\UseMicrotypeSet[protrusion]{basicmath} % disable protrusion for tt fonts
}{}
\PassOptionsToPackage{hyphens}{url} % url is loaded by hyperref
\usepackage[unicode=true]{hyperref}
\hypersetup{
            pdftitle={Capacity},
            pdfauthor={Andy Borst},
            pdfborder={0 0 0},
            breaklinks=true}
\urlstyle{same}  % don't use monospace font for urls
\usepackage[margin=1in]{geometry}
\usepackage{color}
\usepackage{fancyvrb}
\newcommand{\VerbBar}{|}
\newcommand{\VERB}{\Verb[commandchars=\\\{\}]}
\DefineVerbatimEnvironment{Highlighting}{Verbatim}{commandchars=\\\{\}}
% Add ',fontsize=\small' for more characters per line
\usepackage{framed}
\definecolor{shadecolor}{RGB}{248,248,248}
\newenvironment{Shaded}{\begin{snugshade}}{\end{snugshade}}
\newcommand{\KeywordTok}[1]{\textcolor[rgb]{0.13,0.29,0.53}{\textbf{#1}}}
\newcommand{\DataTypeTok}[1]{\textcolor[rgb]{0.13,0.29,0.53}{#1}}
\newcommand{\DecValTok}[1]{\textcolor[rgb]{0.00,0.00,0.81}{#1}}
\newcommand{\BaseNTok}[1]{\textcolor[rgb]{0.00,0.00,0.81}{#1}}
\newcommand{\FloatTok}[1]{\textcolor[rgb]{0.00,0.00,0.81}{#1}}
\newcommand{\ConstantTok}[1]{\textcolor[rgb]{0.00,0.00,0.00}{#1}}
\newcommand{\CharTok}[1]{\textcolor[rgb]{0.31,0.60,0.02}{#1}}
\newcommand{\SpecialCharTok}[1]{\textcolor[rgb]{0.00,0.00,0.00}{#1}}
\newcommand{\StringTok}[1]{\textcolor[rgb]{0.31,0.60,0.02}{#1}}
\newcommand{\VerbatimStringTok}[1]{\textcolor[rgb]{0.31,0.60,0.02}{#1}}
\newcommand{\SpecialStringTok}[1]{\textcolor[rgb]{0.31,0.60,0.02}{#1}}
\newcommand{\ImportTok}[1]{#1}
\newcommand{\CommentTok}[1]{\textcolor[rgb]{0.56,0.35,0.01}{\textit{#1}}}
\newcommand{\DocumentationTok}[1]{\textcolor[rgb]{0.56,0.35,0.01}{\textbf{\textit{#1}}}}
\newcommand{\AnnotationTok}[1]{\textcolor[rgb]{0.56,0.35,0.01}{\textbf{\textit{#1}}}}
\newcommand{\CommentVarTok}[1]{\textcolor[rgb]{0.56,0.35,0.01}{\textbf{\textit{#1}}}}
\newcommand{\OtherTok}[1]{\textcolor[rgb]{0.56,0.35,0.01}{#1}}
\newcommand{\FunctionTok}[1]{\textcolor[rgb]{0.00,0.00,0.00}{#1}}
\newcommand{\VariableTok}[1]{\textcolor[rgb]{0.00,0.00,0.00}{#1}}
\newcommand{\ControlFlowTok}[1]{\textcolor[rgb]{0.13,0.29,0.53}{\textbf{#1}}}
\newcommand{\OperatorTok}[1]{\textcolor[rgb]{0.81,0.36,0.00}{\textbf{#1}}}
\newcommand{\BuiltInTok}[1]{#1}
\newcommand{\ExtensionTok}[1]{#1}
\newcommand{\PreprocessorTok}[1]{\textcolor[rgb]{0.56,0.35,0.01}{\textit{#1}}}
\newcommand{\AttributeTok}[1]{\textcolor[rgb]{0.77,0.63,0.00}{#1}}
\newcommand{\RegionMarkerTok}[1]{#1}
\newcommand{\InformationTok}[1]{\textcolor[rgb]{0.56,0.35,0.01}{\textbf{\textit{#1}}}}
\newcommand{\WarningTok}[1]{\textcolor[rgb]{0.56,0.35,0.01}{\textbf{\textit{#1}}}}
\newcommand{\AlertTok}[1]{\textcolor[rgb]{0.94,0.16,0.16}{#1}}
\newcommand{\ErrorTok}[1]{\textcolor[rgb]{0.64,0.00,0.00}{\textbf{#1}}}
\newcommand{\NormalTok}[1]{#1}
\usepackage{graphicx,grffile}
\makeatletter
\def\maxwidth{\ifdim\Gin@nat@width>\linewidth\linewidth\else\Gin@nat@width\fi}
\def\maxheight{\ifdim\Gin@nat@height>\textheight\textheight\else\Gin@nat@height\fi}
\makeatother
% Scale images if necessary, so that they will not overflow the page
% margins by default, and it is still possible to overwrite the defaults
% using explicit options in \includegraphics[width, height, ...]{}
\setkeys{Gin}{width=\maxwidth,height=\maxheight,keepaspectratio}
\IfFileExists{parskip.sty}{%
\usepackage{parskip}
}{% else
\setlength{\parindent}{0pt}
\setlength{\parskip}{6pt plus 2pt minus 1pt}
}
\setlength{\emergencystretch}{3em}  % prevent overfull lines
\providecommand{\tightlist}{%
  \setlength{\itemsep}{0pt}\setlength{\parskip}{0pt}}
\setcounter{secnumdepth}{0}
% Redefines (sub)paragraphs to behave more like sections
\ifx\paragraph\undefined\else
\let\oldparagraph\paragraph
\renewcommand{\paragraph}[1]{\oldparagraph{#1}\mbox{}}
\fi
\ifx\subparagraph\undefined\else
\let\oldsubparagraph\subparagraph
\renewcommand{\subparagraph}[1]{\oldsubparagraph{#1}\mbox{}}
\fi

% set default figure placement to htbp
\makeatletter
\def\fps@figure{htbp}
\makeatother


\title{Capacity}
\author{Andy Borst}
\date{September 4, 2020}

\begin{document}
\maketitle

\subsection{R Markdown}\label{r-markdown}

This is first draft of capicity.

\begin{Shaded}
\begin{Highlighting}[]
\KeywordTok{library}\NormalTok{(tidyverse)}
\end{Highlighting}
\end{Shaded}

\begin{verbatim}
## -- Attaching packages -------------------------------------------------------------- tidyverse 1.3.0 --
\end{verbatim}

\begin{verbatim}
## v ggplot2 3.3.2     v purrr   0.3.4
## v tibble  3.0.3     v dplyr   1.0.0
## v tidyr   1.1.0     v stringr 1.4.0
## v readr   1.3.1     v forcats 0.5.0
\end{verbatim}

\begin{verbatim}
## -- Conflicts ----------------------------------------------------------------- tidyverse_conflicts() --
## x dplyr::filter() masks stats::filter()
## x dplyr::lag()    masks stats::lag()
\end{verbatim}

\begin{Shaded}
\begin{Highlighting}[]
\KeywordTok{library}\NormalTok{(tidymodels)}
\end{Highlighting}
\end{Shaded}

\begin{verbatim}
## -- Attaching packages ------------------------------------------------------------- tidymodels 0.1.1 --
\end{verbatim}

\begin{verbatim}
## v broom     0.7.0      v recipes   0.1.13
## v dials     0.0.8      v rsample   0.0.7 
## v infer     0.5.3      v tune      0.1.1 
## v modeldata 0.0.2      v workflows 0.1.2 
## v parsnip   0.1.2      v yardstick 0.0.7
\end{verbatim}

\begin{verbatim}
## -- Conflicts ---------------------------------------------------------------- tidymodels_conflicts() --
## x scales::discard() masks purrr::discard()
## x dplyr::filter()   masks stats::filter()
## x recipes::fixed()  masks stringr::fixed()
## x dplyr::lag()      masks stats::lag()
## x yardstick::spec() masks readr::spec()
## x recipes::step()   masks stats::step()
\end{verbatim}

\begin{Shaded}
\begin{Highlighting}[]
\KeywordTok{library}\NormalTok{(skimr)}
\KeywordTok{library}\NormalTok{(lubridate)}
\end{Highlighting}
\end{Shaded}

\begin{verbatim}
## 
## Attaching package: 'lubridate'
\end{verbatim}

\begin{verbatim}
## The following objects are masked from 'package:base':
## 
##     date, intersect, setdiff, union
\end{verbatim}

\begin{Shaded}
\begin{Highlighting}[]
\KeywordTok{source}\NormalTok{(}\DataTypeTok{file =} \StringTok{"Data_Access/database_functions.R"}\NormalTok{)}

\NormalTok{conn <-}\StringTok{ }\KeywordTok{sql01_con}\NormalTok{(}\StringTok{"Coin"}\NormalTok{)}
\end{Highlighting}
\end{Shaded}

\begin{verbatim}
## 
## Attaching package: 'jsonlite'
\end{verbatim}

\begin{verbatim}
## The following object is masked from 'package:purrr':
## 
##     flatten
\end{verbatim}

\begin{Shaded}
\begin{Highlighting}[]
\KeywordTok{db_list_tables}\NormalTok{(conn) }\OperatorTok\StringTok{ }\NormalTok{.[}\KeywordTok{matches}\NormalTok{(}\StringTok{"eek"}\NormalTok{,}\DataTypeTok{vars=}\NormalTok{.)]}
\end{Highlighting}
\end{Shaded}

\begin{verbatim}
## [1] "Weeks"
\end{verbatim}

\begin{Shaded}
\begin{Highlighting}[]
\CommentTok{#   }
\CommentTok{# Weeks <- tbl(conn,"Weeks") %>%}
\CommentTok{#   collect()}
\NormalTok{LineMap <-}\StringTok{ }\KeywordTok{tbl}\NormalTok{(conn,}\StringTok{"ddfJobType"}\NormalTok{) }\OperatorTok
\StringTok{  }\KeywordTok{select}\NormalTok{(JobPrefix, ProductLine) }\OperatorTok
\StringTok{  }\KeywordTok{mutate}\NormalTok{(}\DataTypeTok{JobPrefix =} \KeywordTok{str_trim}\NormalTok{(JobPrefix)) }\OperatorTok\StringTok{ }
\StringTok{  }\KeywordTok{filter}\NormalTok{(}\OperatorTok{!}\KeywordTok{is.na}\NormalTok{(ProductLine)) }\OperatorTok\StringTok{ }
\StringTok{  }\KeywordTok{collect}\NormalTok{()}

\NormalTok{weekRange <-}\StringTok{ }\KeywordTok{c}\NormalTok{(}\KeywordTok{seq}\NormalTok{(}\KeywordTok{isoweek}\NormalTok{(}\KeywordTok{today}\NormalTok{()), }\KeywordTok{isoweek}\NormalTok{(}\KeywordTok{today}\NormalTok{()) }\OperatorTok{+}\StringTok{ }\DecValTok{8}\NormalTok{), }\DecValTok{99}\NormalTok{)}

\NormalTok{vw <-}\StringTok{ }\KeywordTok{tbl}\NormalTok{(conn,}\StringTok{"vwReportKeyData"}\NormalTok{) }\OperatorTok\StringTok{ }
\StringTok{  }\KeywordTok{select}\NormalTok{(Brand, ScheduledCompleteDate, Plant, ReceivedDate, OrderCaseTotal, ShopFloorNumber, OrderTotal) }\OperatorTok\StringTok{ }
\StringTok{  }\KeywordTok{collect}\NormalTok{()}

\NormalTok{t <-}\StringTok{ }\NormalTok{vw }\OperatorTok
\StringTok{  }\KeywordTok{filter}\NormalTok{(}\OperatorTok{!}\KeywordTok{is.na}\NormalTok{(ReceivedDate))  }\OperatorTok\StringTok{ }
\StringTok{  }\KeywordTok{mutate}\NormalTok{(}\DataTypeTok{JobPrefix =} \KeywordTok{str_sub}\NormalTok{(ShopFloorNumber, }\DecValTok{1}\NormalTok{, }\DecValTok{2}\NormalTok{)) }\OperatorTok
\StringTok{  }\KeywordTok{mutate}\NormalTok{(}\DataTypeTok{weekNo =} \KeywordTok{if_else}\NormalTok{(}\KeywordTok{year}\NormalTok{(ScheduledCompleteDate) }\OperatorTok{==}\StringTok{ }\DecValTok{2001}\NormalTok{, }\DecValTok{99}\NormalTok{, }\KeywordTok{isoweek}\NormalTok{(ScheduledCompleteDate))) }
  

\NormalTok{g <-}\StringTok{ }\KeywordTok{inner_join}\NormalTok{(t,LineMap,}\DataTypeTok{by=}\StringTok{"JobPrefix"}\NormalTok{) }\OperatorTok\StringTok{ }
\StringTok{  }\KeywordTok{select}\NormalTok{(weekNo, OrderCaseTotal,OrderTotal, Brand, Plant, ProductLine) }\OperatorTok\StringTok{ }
\StringTok{  }\KeywordTok{filter}\NormalTok{(weekNo }\OperatorTok\StringTok{ }\NormalTok{weekRange) }\OperatorTok\StringTok{ }
\StringTok{  }\KeywordTok{group_by}\NormalTok{(weekNo, Brand, Plant, ProductLine) }\OperatorTok\StringTok{ }
\StringTok{  }\KeywordTok{summarise}\NormalTok{(}\DataTypeTok{Cases =} \KeywordTok{sum}\NormalTok{(OrderCaseTotal), }\DataTypeTok{Dollars =} \KeywordTok{sum}\NormalTok{(OrderTotal)) }\OperatorTok\StringTok{ }
\StringTok{  }\KeywordTok{ungroup}\NormalTok{()}
\end{Highlighting}
\end{Shaded}

\begin{verbatim}
## `summarise()` regrouping output by 'weekNo', 'Brand', 'Plant' (override with `.groups` argument)
\end{verbatim}

\begin{Shaded}
\begin{Highlighting}[]
\NormalTok{g }\OperatorTok\StringTok{ }
\StringTok{  }\KeywordTok{filter}\NormalTok{(Plant }\OperatorTok{==}\StringTok{ "Elkins"}\NormalTok{) }\OperatorTok\StringTok{ }
\StringTok{  }\KeywordTok{pivot_wider}\NormalTok{(}\DataTypeTok{names_from =} \KeywordTok{c}\NormalTok{(}\StringTok{"Brand"}\NormalTok{, }\StringTok{"ProductLine"}\NormalTok{), }\DataTypeTok{values_from =} \KeywordTok{c}\NormalTok{(}\StringTok{"Cases"}\NormalTok{, }\StringTok{"Dollars"}\NormalTok{))}
\end{Highlighting}
\end{Shaded}

\begin{verbatim}
## # A tibble: 8 x 12
##   weekNo Plant Cases_Greenfiel~ `Cases_Greenfie~ `Cases_Siteline~
##    <dbl> <chr>            <int>            <int>            <int>
## 1     36 Elki~              135               75              166
## 2     37 Elki~               48               70              168
## 3     38 Elki~              133              101              137
## 4     39 Elki~               93               50              239
## 5     40 Elki~              105               46              219
## 6     41 Elki~               NA               24              383
## 7     42 Elki~               NA               NA              156
## 8     99 Elki~               NA               NA               NA
## # ... with 7 more variables: `Cases_Siteline_Sales Aid` <int>,
## #   `Cases_Greenfield_Sales Aid` <int>, Dollars_Greenfield_Framed <dbl>,
## #   `Dollars_Greenfield_Full Access` <dbl>, `Dollars_Siteline_Full
## #   Access` <dbl>, `Dollars_Siteline_Sales Aid` <dbl>,
## #   `Dollars_Greenfield_Sales Aid` <dbl>
\end{verbatim}

\subsection{Including Plots}\label{including-plots}

You can also embed plots, for example:

\includegraphics{Capacity_files/figure-latex/pressure-1.pdf}

Note that the \texttt{echo\ =\ FALSE} parameter was added to the code
chunk to prevent printing of the R code that generated the plot.

\end{document}
